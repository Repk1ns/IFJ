\documentclass[a4paper,11pt]{article}
\usepackage[left=2cm,text={17cm, 24cm},top=3cm]{geometry} 
\usepackage{times}
\usepackage[czech]{babel}
\usepackage[utf8]{inputenc}
\bibliographystyle{czechiso}
\pagenumbering{arabic}

\begin{document}
	\begin{titlepage}
		\begin{center} \Huge \textsc{Vysoké učení technické v Brně \\ \huge Fakulta informačních technologií\\} 
			\vspace{\stretch{0.382}} \LARGE Formální jazyky a překladače \\ \Huge Implementace překladače imperativního jazyka IFJ19 \vspace{\stretch{0.618}} 
			
			\end{center}	

		{\LARGE \today}
	\end{titlepage}
	\newpage
	\tableofcontents
    \newpage


\section{Úvod}
Cílem projektu je tvorba programu v jazyce C, který  je podmnožinou jazyka Python 3. Načítá zdrojový kód zapsaný ve zdrojovém jazyce IFJ19 a překládá jej do cílového jazyka IFJcode19 (mezikód).

\section{Návrh a~implementace}
Projekt se skládá ze čtyř hlavních částí. 

\subsection{Lexikální analýza}
Tato část, také jinak nazývaná scanner, je první implementovanou částí. Hlavní funkcí je \texttt{getNextSymbol}, která čte ze vstupu všechny příchozí znaky.

\section{Závěr}



\newpage
\bibliography{ref}
\end{document}
